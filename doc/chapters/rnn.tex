\chapter{ECG prediction using a Recurrent Neural Network}\label{ch:rnn}

In this chapter I will discuss the design and implementation of a Long
Short-Term Memory (LSTM) Recurrent Neural Network (RNN) that is able to predict
the next value of the ECG signal given as input the ECG signal up to the
previous time step and some other signals. This implements the requirements for
the Task 4.2 of the Project Specifications.

The approach followed to develop the RNN is similar to the one used in
\chref{ch:rnn}. In particular, since training of CNNs and RNNs is similar, the
script \code{rnntrain.m} is a copy-pasted and adapted version of the
\code{cnntrain.m} script.

Note that, for this task, I've changed again the way data augmentation is
performed. Here, I'll extract a total of \(60000\) samples of \(61\) time
steps, since \(6000\) samples were not sufficient to train the RNN. Moreover,
the \code{fixdata} stage is instructed to divide samples whenever a hole larger
than 2 milliseconds (the fastest sampling rate) is found in the data: to train
an RNN, I do not want to have any hole at all in the data.

The developed RNN uses 8 signals of the 11 signals available in the dataset.
The 8 signals selected are all the signals that are sampled with a sampling
rate of \(500\) Hz (the \code{pleth\_*} signals, the \code{temp\_3} signal
and, obiously, the \code{ecg} signal). This choice has been made in order to
avoid to use signals sampled at lower rates, which remain constants for some
time steps and may degrade the performance of the network.

\section{Selection of the size of the window}\label{sec:rnnwinsize}

The starting LSTM RNN architecture is shown in \lstref{lst:startingrnn}.

\lstinputlisting[language={matlab}, label={lst:startingrnn},
style={Matlab-editor}, basicstyle={\footnotesize\ttfamily}, caption={Starting
architecture for the LSTM.}]{startingrnn.m}

RNN definitions from 1 to 10 are used to find the optimal window size, using
\(50\) as the number of neurons and training for \(50\) epochs. Results are
shown in \tableref{table:rnnwinsize}. A window size of \(10\) is a valid
choice, based on the RMSE on the best loss iteration.

\begin{table}[hbtp]
	\centering
	\begin{tabular}{|c|c|c|c|}
		\toprule
		\# & Window size & RMSE & Best RMSE \\
		\midrule
		1 & 10 & \(6128.04\) & \(3087.69\) \\
		2 & 20 & \(8666.25\) & \(8949.86\) \\
		3 & 30 & \(4839.1\) & \(5496.51\) \\
		4 & 40 & \(7023.05\) & \(5328.75\) \\
		5 & 50 & \(7434.16\) & \(6413.42\) \\
		6 & 60 & \(9297.86\) & \(8541.49\) \\
		7 & 12 & \(5614.25\) & \(5315.68\) \\
		8 & 15 & \(7863.83\) & \(6274.77\) \\
		9 & 9 & \(7406.52\) & \(4481.75\) \\
		10 & 8 & \(8528.44\) & \(6514.86\) \\
		\bottomrule
	\end{tabular}
	\caption{Trying different window sizes for the
	RNN.}\label{table:rnnwinsize}
\end{table}

\section{Selection of the normalization algorithm}\label{sec:rnnnormalization}

I've tried all possible choices for the normalization algorithm for the input
sequences. Results in \tableref{table:rnnnormalization}.

\begin{table}[hbtp]
	\centering
	\begin{tabular}{|c|c|c|c|}
		\toprule
		\# & Normalization algorithm & RMSE & Best RMSE \\
		\midrule
		1 & \code{rescale-symmetric} & \(6128.04\) & \(3087.69\) \\
		11 & \code{zerocenter} & \(13938.16\) & \(11285.47\) \\
		12 & \code{zscore} & \(3157.22\) & \(2930.36\) \\
		13 & \code{rescale-zero-one} & \(10997.16\) & \(10982.06\) \\
		\bottomrule
	\end{tabular}
	\caption{Changing the normalization algorithm for the
	RNN.}\label{table:rnnnormalization}
\end{table}

Best results were obtained using \code{zscore}.

\section{Selection of the number of neurons}\label{sec:rnnneurons}

RNN definitions from 14 to 20 are used to try different values for the number
of neurons in the LSTM layer. Results are shown in \tableref{table:rnnneurons}.
Best results are obtained using \(65\) neurons.

\begin{table}[hbtp]
	\centering
	\begin{tabular}{|c|c|c|c|}
		\toprule
		\# & \# of neurons & RMSE & Best RMSE \\
		\midrule
		12 & 50 & \(3157.22\) & \(2930.36\) \\
		14 & 20 & \(15698.21\) & \(13928.67\) \\
		15 & 25 & \(10191.53\) & \(10191.53\) \\
		16 & 30 & \(9127.55\) & \(7290.41\) \\
		17 & 40 & \(6569.25\) & \(2298.45\) \\
		18 & 65 & \(3389.88\) & \(1550.45\) \\
		19 & 80 & \(15068.95\) & \(3021.44\) \\
		20 & 100 & \(4208.14\) & \(2406.95\) \\
		\bottomrule
	\end{tabular}
	\caption{Changing the number of neurons for the LSTM
	layer.}\label{table:rnnneurons}
\end{table}

\section{Selection of training algorithm and options}\label{sec:rnntraining}

RNN definitions 21 and 22 are used to try training algorithms different from
the Adam solver. Results are shown in \tableref{table:rnntraining}. Excellent
performances are obtain with the RMSProp optimizer (\(R = 0.981\) in just 50
epochs).

\begin{table}[hbtp]
	\centering
	\begin{tabular}{|c|c|c|c|}
		\toprule
		\# &  & RMSE & Best RMSE \\
		\midrule
		18 & Adam & \(3389.88\) & \(1550.45\) \\
		21 & SGDM & \(23551.03\) & \(22790.77\) \\
		22 & RMSProp & \(737.76\) & \(613.98\) \\
		\bottomrule
	\end{tabular}
	\caption{Trying different optimizers.}\label{table:rnntraining}
\end{table}

Chosen training options are the following:
\begin{description}
	\item[MiniBatchSize] \(300\). I've tried some values and this seems to
		be a good one.
	\item[Shuffle] every epoch.
	\item[InitialLearnRate] \(0.1\). In an attempt to speed up the
		training, since it requires a huge amount of epochs to reach
		good results.
	\item[LearnRateSchedule] \code{piecewise}.
	\item[LearnRateDropFactor] \(0.4\). To not reduce the learning rate too
		much, otherwise too epochs are needed to train the network.
	\item[LearnRateDropPeriod] every \(50\) epochs.
\end{description}

\section{Final Long Short-Term Memory Recurrent Neural
Network}\label{sec:rnnfinalrnn}

The final RNN is shown in \lstref{lst:finalrnn}.

\lstinputlisting[language={matlab}, label={lst:finalrnn},
style={Matlab-editor}, basicstyle={\footnotesize\ttfamily}, caption={The final
RNN architecture and training options.}]{rnndef.m}

This network has been trained for \(350\) epochs. Training ended after
\(33337\) iterations (\(238\) epochs plus some iterations), due to validation
loss not improving anymore. The network achieved excellent results with \(R >
0.99\) for all sets. Complete results are shown in \lstref{lst:rnnresults}.
\figref{fig:rnnregression} shows the regression plots for the RNN.
\figref{fig:rnnpredictions} shows an example of how the RNN predicts values for
the ECG.

\lstinputlisting[language={}, label={lst:rnnresults},
caption={Results of the final Long Short-Term Memory Recurrent Neural
Network.}]{rnnresults.txt}

\begin{figure}[htbp]
	\centering
	\includegraphics[width=\textwidth]{rnnregression}
	\caption{Regression plots for all three sets for the LSTM RNN
	that predicts values for the ECG.}\label{fig:rnnregression}
\end{figure}

\begin{figure}[htbp]
	\centering
	\includegraphics[width=\textwidth]{rnnpredictions}
	\caption{The 2 lines nearly overlap each other, proving that the RNN
	performs very well (zoom: this is vector
	graphics).}\label{fig:rnnpredictions}
\end{figure}

