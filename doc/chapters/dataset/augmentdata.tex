\section{Data augmentation}\label{sec:augmentdata}

The provided data is not sufficient to train the intelligent systems required
by the Project Specifications: there are only 22 subjects recorded during 3
different activities, so there are only a total of 66 recordings available
(plus one more, after the \texttt{fixdata} stage).

Data augmentation is performed by the \texttt{augmentdata} stage. I considered
several methods for data augmentation:
\begin{enumerate}
\item Data augmentation by \standout{adding some noise or distorsions} to
	available signals.
\item \emph{Feature} augmentation using an \standout{autoencoder}.
\item Data augmentation via a \standout{(Conditional) (Wasserstein) Generative
	Adversarial Network} (CW-GAN).
\item Data augmentation via \standout{random subsampling}.
\end{enumerate}

The first method was discarded because I need to extract several dozen of data
from each recording to have a fair amount of data for the purposes of this
project: the dataset after the augmentation would be composed of different
groups of similar data, and the intelligent systems would probably learn to
predict the group of the data instead of generalizing, leading to overfitting.

The second method has the same issue: the autoencoder would create various
feature matrices from the same recordings, creating different groups of feature
matrices with similar statistical characteristics.

The third method is too complex, and would probably require a lot of time to
generate the new signals. Moreover, it is too hard to determine \emph{a priori}
the quality of the generated signals via this method. Generated signals may be
too artificial to be used to represent real world signals: it's a risky
approach.

I chose to go with the forth method. Extracting random samples from the
more-than-8-minutes-long recordings is a simple and secure approach: data is
real, not artificial, and even if the intelligent systems developed in this
project are not tied to a specific purpose, using more that 8 minutes of data
to estimate ECG/activity is probably too much for any purpose (I would expect
an intelligent system to be able to ``say something'' in less than 8 minutes).

So, the \texttt{augmentdata} stage extracts several random subsamples from the
original 66 (67) samples. Each subsample has a variable duration between
25 and 40 seconds (this is done in order to let the intelligent systems to
generalize over the duration of the signals, so they can learn to estimate
ECG/activity independently from the exact duration of the signal), and 2000
samples are extracted \emph{from each activity}, for a total of \standout{6000
samples} available after data augmentation.
