\section{Features and targets extraction}\label{sec:extractfeaturestargets}

In this section, I will describe in details how the features and targets
extraction is implemented.

\subsection{Defining features}\label{subsec:getfeaturesstage}

The first step is to define a list of features to extract.

I defined an initial list of 264 features, composed of 24 statistical features
extracted from each of the 11 input signals available. For each input signal,
stage \texttt{getfeatures} defines the following features:

\noindent\underline{\standout{Time domain:}}
\begin{multicols}{2}
\begin{itemize}
\item Mean
\item Median
\item Range
\item Minimum value
\item Maximum value
\item Variance
\item Inter-quartile range (IQR)
\item Mean of the difference between consecutive data points
\item Kurtosis\footnote{Kurtosis is a measure of the ``tailedness'' of a
	distribution (how often outliers occcur). It is computed as the fourth
	central moment divided by the square of the variance (the fourth power
	of the standard deviation).}
\item Skewness\footnote{Skewness is a measure of the asymmetry of a
	distribution. It is computed as the third central moment divided by the
	cube of the standard deviation.}
\item Root-mean-square (RMS)
\item Ratio of largest absolute to root mean squared value (\code{peak2rms})
\item Harmonic mean
\item Mean absolute deviation. It is \code{mean(abs(x - mean(x)))}.
\item Range of the cumulative sum
\item Mode
\item Trapezoidal integration
\end{itemize}
\end{multicols}

\noindent\underline{\standout{Frequency domain:}}
\begin{multicols}{2}
\begin{itemize}
\item Mean frequency
\item Median frequency
\item Occupied bandwidth
\item Lowest frequency
\item Highest frequency
\item Power within the occupied bandwidth
\item Half-power bandwidth (3 dB bandwidth of the power spectrum)
\end{itemize}
\end{multicols}

Note that some features depends on other features of the list \exgratia{the
\emph{range} is computed as \(max - min\), which are also part of the list ---
same applies to the occupied bandwith which is computed as the difference of
highest and lowest frequencies}: the exact features to use will be selected
later in \secref{sec:featureselection}, also by removing correlated/dependent
features.

The list of features generated by the \texttt{getfeatures} stage is a cell
array of character vectors built as follow:
\begin{verbatim}
pleth_1:mean
pleth_1:median
pleth_1:range
...
lc_2:meandiff
lc_2:kurtosis
... (and so on) ...
\end{verbatim}
The first part of the feature name is the name of the input signal, the second
part (after the colon) is the name of the statistical measure extracted from
the signal.

Other possible features have been considered but not included in the list of
features used in this work: the \emph{season of the recording} may be useful
since temperature signals (\texttt{temp\_*}) may be significantly different
during winter and summer, but \emph{all} available data have been extracted on
1\textsuperscript{st} January (so during winter). For the same reason, another
possible feature is the \emph{time of the day}, but \emph{all} signals are
collected more or less at the same time of day. Finally, a good feature is the
\emph{total duration of the signal} (a mean extracted over a signal of some
seconds is very different from the same mean extracted over a signal of some
minutes or hours): I did not included this feature just because I thought about
it too late, after most of the project were implemented and all the data were
collected.

\subsection{Extracting features}\label{subsec:extractfeaturesstage}

The list of features generated by the \texttt{getfeatures} stage is passed to
the \texttt{extractfeatures} stage, along with the augmented dataset from the
\texttt{augmentdata} stage.

The \texttt{extractfeatures} stage also takes as input 2 optional parameters to
define the method to extract the features. In fact, I'm going to extract all
the features using 2 different methods, which I will call ``Normal'' and
``Windowed'':
\begin{description}
\item[Normal] The features are extracted from the entire signal. So, for
	example, \texttt{pleth\_1:mean} is computed as the mean of the entire
	\texttt{pleth\_1} signal.
\item[Windowed] The feature are extracted from 5 \emph{overlapped} windows. So,
	for each feature, we are going to compute 5 values, corresponding to
	the value of the feature computed on each of the 5 windows.
\end{description}

Each feature is then extracted from each input signal, creating a MATLAB
structure with a field for each input signal where each field is a structure
with a field for each statistic extracted containing a \(W \times N\) matrix
where \(W\) is the number of windows and \(N = 6000\) is the number of samples.
For the ``Normal'' method, \(W = 1\). For the ``Windowed'' method, \(W = 5\).
So, for example, in the windowed case, if the resulting MATLAB structure is
called \code{features}, the field \code{features.pleth\_1.mean} will contain a
\(5 \times 6000\) matrix where element of row \(i\) column \(j\) is the value
of the mean computed on window \(i\) for sample \(j\) on signal
\texttt{pleth\_1}.

Note that, even with the ``Windowed'' method, the features on the
\texttt{temp\_3} signal are always extracted in the ``normal way'' \idest{in a
single window}. This is because \texttt{temp\_3} is the ambient temperature: I
do not expect it to change too much rapidly, so there is less information in
extracting it from small windows.

\subsection{Extracting targets}\label{subsec:extracttargetsstage}

The \texttt{extracttargets} stage takes as input the augmented dataset from the
\texttt{augmentdata} stage and extracts 3 vectors: ECG mean, standard deviation
and activity codified as an index (\(sit = 0\), \(walk = 1\), \(run = 2\)).

Note that this stage also takes as input the same 2 optional parameters used in
the \texttt{extractfeatures} stage to define the method to extract the
features. In fact, in the case of the windowed method, the last part of the
signal of each sample may be cut off in order to make every window extracted
from the same sample of the same size (this happens if the total number of
records in a signal is not a multiple of the \code{winStep} computed by the
\texttt{extracttargets} stage).
