\section{Selection of the number of filters}\label{sec:cnnfilters}

Here I will try to increase the number of filters by tuning the \code{nf}
parameter. This parameter is multiplied with the ``depth'' of the convolutional
layer to obtain the number of filters in the layer. Some of the test will use 4
convolutional layers; others will use 5 convolutional layers.

Results in \tableref{table:cnnfilters}. The most complex architecture tested,
with 5 convolutional layers and \code{nf = 64}, is the best one.

\begin{table}[hbtp]
	\centering
	\begin{tabular}{|c|c|c|c|}
		\toprule
		\# & Layers & nf & R \\
		\midrule
		25 & 4 & 16 & \(0.940\) \\
		26 & 5 & 16 & \(0.939\) \\
		27 & 4 & 32 & \(0.944\) \\
		28 & 4 & 48 & \(0.947\) \\
		29 & 4 & 64 & \(0.945\) \\
		30 & 5 & 32 & \(0.946\) \\
		31 & 5 & 48 & \(0.946\) \\
		32 & 5 & 64 & \(0.949\) \\
		\bottomrule
	\end{tabular}
	\caption{Increasing the number of filters. Only the \(R\) parameter is
	reported.}\label{table:cnnfilters}
\end{table}
