\section{Selection of the number of convolutional
layers}\label{sec:cnnconvlayers}

CNN definitions from 23 to 26 try different architectures by increasing the
number of convolutional blocks. Each block has the same structure, as defined
in \secref{subsec:cnnfinalbase}. Note that now the number of filters of a
convolutional layer is computed as \(i\times16\), where \(i\) is an integer
which increases going down in the CNN, so that the first convolutional layer
has 16 filters; the second has 32 filters; the third has 48 filters and so on.
As we move forward through layers, the patterns that the convolutional layers
must try to identify become more complex and more filters may allow the network
to better find these patterns.

To account for the increased number of filters in the last convolutional layer,
I've also increased the size of fully connected layers. Since now the
architecture is quite complex, I am also going to perform tests using 60 epochs
(instead of 30, as before).

Results in \tableref{table:cnnconvlayers}. The best results are given by the
last 2 CNNs, with 4 and 5 convolutional blocks.

\begin{table}[hbtp]
	\centering
	\begin{tabular}{|c|c|c|c|c|}
		\toprule
		\# & Layers & RMSE & Best RMSE & R \\
		\midrule
		7 (30 epochs) & 1 & \(2013.3\) & \(514.46\) & \(0.748\) \\
		7\emph{bis} (30 epochs) & 1 & \(680.05\) & \(655.85\) & \(0.757\) \\
		23 & 2 & \(467.33\) & \(413.65\) & \(0.915\) \\
		24 & 3 & \(466.06\) & \(335.28\) & \(0.935\) \\
		25 & 4 & \(382.39\) & \(406.14\) & \(0.940\) \\
		26 & 5 & \(500.22\) & \(243.24\) & \(0.939\) \\
		\bottomrule
	\end{tabular}
	\caption{Increasing the number of convolutional
	blocks.}\label{table:cnnconvlayers}
\end{table}
