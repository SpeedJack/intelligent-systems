\section{Feature selection}\label{sec:fuzzyfeatureselection}

In order to select the best 3 features used for the classifier developed in
\chref{ch:activity}, \code{sequentialfs} with \standout{backward} feature
selection has been run on the list of features used for the MLP. The function
evaluated by \code{sequentialfs} uses the architecture of the MLP of the
previous chapter, but it changes proportionally the number of hidden neurons in
both layers as the number of inputs goes down.

The task is performed by the script \code{featureselection\_fuzzy.m}, which
uses the stage \code{selectfeatures\_fuzzy} (adapted, for this task, from the
\code{selectfeatures} stage used for the MLPs).

The following are the 3 best features selected by \code{sequentialfs}:
\begin{itemize}
	\item \code{lc\_1:mean} windowed.
	\item \code{pleth\_1:mean} \emph{not} windowed.
	\item \code{pleth\_2:mean} windowed.
\end{itemize}

Note that, even if 2 of the 3 features selected are extracted with the windowed
approach, in the following I'm going to extract them in a single window,
otherwise the FIS would have 11 input variables instead of 3.
